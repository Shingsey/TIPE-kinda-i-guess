\documentclass[a4paper, 100pt]{article}
\usepackage[utf8]{inputenc}
\usepackage{amsmath}
\usepackage{amssymb}
\begin{document}
\title{\textbf{Groupe de tresse et chiffrement}}
\author{Axel De Masure}
\date{12 janvier 2021}
\maketitle


L'utilisation des groupes de tresse dans des systèmes de chiffrement est un domaine de recherche actif depuis les années 2000, champ qui a rapidement rencontré des difficultés, tout les systèmes jusqu'alors proposé ont été attaqué (avec brio). Malheureusement, les groupes de tresses étant assez peu connus des cryptologue, peu de tentatives ont à nouveau vu le jour.
\section{Introduction aux groupes de tresse}
La théorie des tresses est une branche de la topologie algébrique, proche voisine de la théorie des nœuds. Les objets manipulés sont des tresses, composés d'un nombre de brins finis qui s'entremêlent d'une extremité à l'autre (insérer figure).
Naturellement ses objets sont étudiés sous l'œil topologique : si on peut représenter une même tresse d'une infinité de manière différente, seuls les nœuds fondamentaux nous intéresse.
On retrouve alors une structure de groupe que l'on va détailler ci-dessous.


On fixe $ n \in \mathbb{N} $ notre nombre de brins. Le sens de lecture des tresses est l'horizontale de gauche à droite.
   On dispose alors d'ensembles de générateurs pour représenter nos tresses. L'un d'entre eux est l'utilisation des

\end{document}
